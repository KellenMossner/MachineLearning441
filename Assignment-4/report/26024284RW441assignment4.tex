\documentclass[conference]{IEEEtran}

%Template version as of 6/27/2024

\usepackage{cite}
\usepackage{amsmath,amssymb,amsfonts}
\usepackage{algorithm}
\usepackage{algorithmicx}
\usepackage{algpseudocode}
\usepackage{graphicx}
\usepackage{textcomp}
\usepackage{xcolor}
\usepackage{array}
\usepackage{multirow}
\usepackage{url}
\usepackage[caption=false,font=footnotesize]{subfig}
\newcolumntype{L}[1]{>{\raggedright\arraybackslash}p{#1}}
\def\BibTeX{{\rm B\kern-.05em{\sc i\kern-.025em b}\kern-.08em
		T\kern-.1667em\lower.7ex\hbox{E}\kern-.125emX}}
\begin{document}
	
	\title{Machine Learning 441 Assignment 4: Isolation Forests}
	\author{\IEEEauthorblockN{KT M\"ossner}
		\IEEEauthorblockA{26024284 \\
			\textit{Computer Science Department} \\
			\textit{Stellenbosch University} \\
			26024284@sun.ac.za}
	}
	\maketitle
	
	\begin{abstract}
	\end{abstract}
	
	\section{Introduction}
	The introduction sets the stage for the remainder of your report. You usually have very general statements here.
	The introduction prepares the reader for what to expect from reading your report. In general, the introduction
	should either contain or be a summary of your ENTIRE report.
	
	This report is based on the implementation provided in \cite{github}, which contains the source code for Assignment 4.
	
	\section{Background}\label{B}
	A very high level discussion on the problem domain and the algorithms and/or approaches that you have used.
	Do not be too specific on the algorithms and approaches. This section is typically where the “base cases” of
	concepts that appear throughout the remainder of your report are discussed. It is also an ideal place to refer
	a reader to other sources containing relevant information on the topic but which is outside the scope of your
	assignment. It is the perfect place for pseudo code. Remember to discuss very generally. After reading this
	section the marker should be able to determine whether or not you know what you’re talking about.
	
	\section{Implementation}\label{I}
	In this section you discuss how you approached, implemented and solved your assignment choice. You provide
	pseudo code where necessary and discussions of the solutions that you have implemented. This is also the
	section where your discussion specializes on the concepts mentioned in the background section. Be very specific
	in your discussions in this section.
	
	\section{Empirical Process}\label{EP}
	Here you describe the empirical procedure followed to apply your algorithms to obtain answers to the goals/hy-
	pothesis of the study. You elaborate on the performance measures used and provide the benchmark problems
	used. Provide all control parameter values with a motivation for why you have used these, and state the number
	of independent runs. After reading this section (in addition to the background) the reader should be able to
	duplicate your experiments to obtain similar results to those obtained by you.
	
	\section{Results \& Discussion}\label{RD}
	This is the section where you report your results obtained from running the experiments as discussed in the
	implementation section. You have to give, at least, averages and standard deviations for the experiments/simu-
	lations. Thoroughly discuss the results that you have obtained and provide clear arguments in support of your
	results and observations from these results. Answer questions like “are these results to be expected?”, “why did
	these results occur?” and “would different circumstances lead to different results?”.
	
	\section{Conclusion}\label{C}
	Very general conclusions about the assignment that you have done. This section “answers” the questions and
	issues that you’ve raised and investigated. This section is, in general, a summary of what you have done, what
	the results where and finally what you concluded from these results. This is the final section in your document
	so be sure that all the issues raised up until now are answered here. This is also the perfect section to discuss
	what you have learnt in doing this assignment.
	
	\begin{thebibliography}{00}
		
		\bibitem{github}
		K. Mossner, ``Assignment-4,'' \textit{GitHub}, 2025. [Online]. Available: \url{https://github.com/KellenMossner/MachineLearning441/tree/main/Assignment-4}. [Accessed: Oct. 1, 2025].
		
		\bibitem{thyroid}
		J. R. Quinlan, ``Induction of decision trees,'' \textit{Machine Learning}, vol. 1, no. 1, pp. 81--106, 1986.
		
		\bibitem{fault}
		M. Buscema, S. Terzi, and W. Tastle. "Steel Plates Faults," UCI Machine Learning Repository, 2010. [Online]. Available: \url{https://doi.org/10.24432/C5J88N}.
		
		\bibitem{satimage2}
		S. Rayana, ``ODDS library,'' Stony Brook University, 2016. [Online]. Available: \url{http://odds.cs.stonybrook.edu}. [Accessed: Oct. 1, 2025].
		
	\end{thebibliography}

	
\end{document}